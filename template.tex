\documentclass
[kulak,english,outline,totalframes]
%%% OPTIONS %%%
%   kulak (default) / kul      -   changes logo
%   english (default) / dutch  -   changes language, passed on to babel
%   outline                    -   adds an outline frame at the start of each section
%   totalframes                -   adds the total number of frames to the footer
%   handout                    -   handout mode, use this when printing slides
{kulakbeamer}

%%% PACKAGES %%%
\usepackage{mathtools,amsthm,amssymb}
\usepackage{tikz-cd}

%%% DATA %%%
\title[Short presentation title]{Actual full-length title which may be very long and not fit on a single line}
%\subtitle{Subtitle}
\author[Short Name]{Long Name} 
\institute[Kulak]{KU Leuven Kulak Kortrijk Campus}
\date{29th April 1967}

\begin{document}
    
    %%% SLIDES %%%
    \section{Introduction}
    
    \begin{frame}{Introduction}
        Welcome to my presentation!
    \end{frame}
    
    \section[Short section title]{Long section title}
    
    \subsection{Subsection title}
    \begin{frame}{Frame title}
        Frame content.
    \end{frame}
    
    \subsection{AMS-LaTeX}
    \begin{frame}{Lefschetz fixed point theorem}
        \begin{theorem}[S. Lefschetz]
            Let \(f \colon X \to X\) be a continuous self-map on a connected, compact polyhedron \(X\).
            If \(L(f) \neq 0\), then \(f\) has at least one fixed point.
        \end{theorem}
    \end{frame}
    
    \subsection{Diagrams}
    \begin{frame}[fragile]{General lifting lemma}
        
        Let \(p \colon Z \to Y\) be a covering map and fix \(y \in Y\) and \(z \in Z\) such that \(p(z) = y\).
        Let \(f \colon X \to Y\) be a continuous map with \(f(x) = y\).
        Suppose that \(Z\) is path-connected and locally path-connected.
        
        \medskip
         
        If
        \begin{equation*}
            f_\pi ( \pi_1(X,x) ) \subseteq p_\pi(\pi_1(Z,z)),
        \end{equation*}
        then there exists a continuous map \(\tilde{f} \colon X \to Z\) such that:
        \begin{itemize}
            \item \(p \circ \tilde{f} = f\), i.e.\@ the diagram below commutes,
            \item \(\tilde{f}(x) = z\).
        \end{itemize}
        \[
            % Requires the frame to be given the "fragile" option
            \begin{tikzcd}
                & Z \arrow[d,"p"]\\
                X \arrow[ur,"\tilde{f}"] \arrow[r,"f"] & Y
            \end{tikzcd}
        \]
    \end{frame}
    
    \section{Conclusion}
    \begin{frame}{Closing frame}
        Make sure to thank your audience and/or ask if there are any questions.
    \end{frame}
    
\end{document}
